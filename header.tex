
%%%%%%%%%%%%%%%%%%%%%%%%%%%%%%%%%%%%%%%%%%%%%%%%%%%%%%%%%%%%%%%%%%%%%%%
%% SSE Thesis Document Template (Thomas Dietrich) %%%%%%%%%%%%%%%%%%%%%
%%%%%%%%%%%%%%%%%%%%%%%%%%%%%%%%%%%%%%%%%%%%%%%%%%%%%%%%%%%%%%%%%%%%%%%

\documentclass[
	12pt,						% font size
	english,					% LANGUAGE english document
	%ngerman,					% LANGUAGE deutsches Dokument, für Umlaute, Silbentrennung etc. (Umlauts, Hyphenation)
	a4paper,					% paper size
	%oneside,					% single-sided document
	twoside,					% two-sided document
	titlepage,					% the title page is used (will start right)
	parskip=half,				% space between paragraphs (half line)
	headings=normal,			% decrease size of headers
	open=right,					% chapter beginnen rechts
	listof=totoc,				% list directories in the table of contents
	bibliography=totoc,			% list bibliography in the table of contents
	captions=tableheading,		% output labeling tables below
	draft=true, 				% draft=true useful during writing, set to false for final document %BUG will stop hyperref from working (bookmarks)
	DIV=13,						%number columns for page layout calculation. The larger, the smaller margins will be. Standard for 12pt is 12
	BCOR=10mm,					%reserved inner margin for binding
	headinclude,				%include header into text area
]{scrreprt}
%%%%%%%%%%%%%%%%%%%%%%%%%%%%%%%%%%%%%%%%%%%%%%%%%%%%%%%%%%%%%%%%%%%%%%%


%% Line Spacing and Margins %%%%%%%%%%%%%%%%%%%%%%%%%%%%%%%%%%%%%%%%%%%
\usepackage[onehalfspacing]{setspace}

%% Scripture and Language %%%%%%%%%%%%%%%%%%%%%%%%%%%%%%%%%%%%%%%%%%%%%
\usepackage[english]{babel}  %LANGUAGE
%\usepackage[ngerman]{babel} %LANGUAGE
\usepackage[T1]{fontenc}
\usepackage[ansinew]{inputenc}
\usepackage{textcomp} % Euro-Sign etc.
\usepackage{lmodern}

%% Page Style %%%%%%%%%%%%%%%%%%%%%%%%%%%%%%%%%%%%%%%%%%%%%%%%%%%%%%%%%
\usepackage[perpage]{footmisc}

% Customize headers and footers
\usepackage[
	automark,						% Create Chapter Information in Header automatically
	headsepline,					% Dividing line under header
	%footsepline,					% Dividing line over footer
	ilines							% Align dividing line left-aligned
]{scrlayer-scrpage}

\pagestyle{scrheadings} % headers and footers
\renewcommand*{\chapterpagestyle}{scrheadings} % header and footer on the first page of chapter
\renewcommand{\headfont}{\normalfont} % font of the header

%TODO: fancyhdr

%% Header
%\ihead{\large{\textsc{\thesistitle}}\\ \small{\thesistitle} \\[2ex] \textit{\headmark}}
%\ihead{\large{\textsc{\thesistype}}\ - \small{\autor} \\ \textit{\headmark}}
\ihead{\textit{\headmark}}
\chead{}
\ohead{}
%\ohead{\includegraphics[scale=0.15]{\logo}}
% broaden header beyond the text
\setheadwidth{textwithmarginpar}

%% Footer
%\ifoot{\small{\thesistype\ \autor}}
\ifoot{}
\cfoot{}
\ofoot{\pagemark}

\recalctypearea %force recalculation of page layout, after setting headers, footers and line spread

%% first tests towards the usage of xetex
%\usepackage[T1]{fontenc}
%\usepackage[utf8x]{inputenc}
%\usepackage{libertine}
%\usepackage[ngerman]{babel}
%\usepackage{xunicode}
%\usepackage{fontspec}
%\usepackage{xltxtra} % For XeLaTeX
%\setromanfont[Mapping=tex-text]{Linux Libertine O}		% serif font
%\setsansfont[Mapping=tex-text]{Linux Biolinum O}		% sanserif font
%\setmonofont[Mapping=tex-text,Scale=0.9]{Courier New}	% font for code
%\setmonofont[Mapping=tex-text]{DejaVu Sans Mono}
%\usepackage{polyglossia}
%\setdefaultlanguage[spelling=new,latesthyphen=true]{english}
%%%%%%%%%%%%%%%%%%%%%%%%%%%%%%%%%%%%%%%%%%%%%%%%%%%%%%%%%%%%%%%%%%%%%%%


%% Global Document Information %%%%%%%%%%%%%%%%%%%%%%%%%%%%%%%%%%%%%%%%
% please fill in your personal information
% do not delete anything here
\newcommand{\thesistitle}{Thesis Title}
\newcommand{\thesistype}{Master Thesis}
\newcommand{\autor}{Author Name}
\newcommand{\mail}{author-email@tu-ilmenau.de}
\newcommand{\courseofstudies}{rocket science}
\newcommand{\matriculationnr}{XXXXX}
\newcommand{\supervisor}{Prof. Responsible}
\newcommand{\secondtutor}{Second Reader}
\newcommand{\faculty}{Department of Computer Science and Automation}
\newcommand{\department}{Systems and Software Engineering Group}
\newcommand{\location}{Ilmenau}
\newcommand{\registrationdate}{XX.~January 20XX}
\newcommand{\submissiondate}{XX.~June 20XX}
\newcommand{\submissiontimestamp}{XX.XX.20XX}
%%%%%%%%%%%%%%%%%%%%%%%%%%%%%%%%%%%%%%%%%%%%%%%%%%%%%%%%%%%%%%%%%%%%%%%


%% Graphics & Illustrations %%%%%%%%%%%%%%%%%%%%%%%%%%%%%%%%%%%%%%%%%%%
\usepackage{graphicx}			% in order to load graphics
\usepackage{wrapfig}			% integration of graphics with wraping text
\usepackage{subfig}				% integration of multiple objects within a float
\usepackage{pdfpages}			% binds a graphic or page (.pdf or .jpg) in the document
\usepackage{float}
\graphicspath{{images/}}
%%%%%%%%%%%%%%%%%%%%%%%%%%%%%%%%%%%%%%%%%%%%%%%%%%%%%%%%%%%%%%%%%%%%%%%


%% Directory Definition %%%%%%%%%%%%%%%%%%%%%%%%%%%%%%%%%%%%%%%%%%%%%%%
%\usepackage{tikz}
%\usetikzlibrary{trees}
%\usepackage{xcolor}
\usepackage{dirtree}
%%%%%%%%%%%%%%%%%%%%%%%%%%%%%%%%%%%%%%%%%%%%%%%%%%%%%%%%%%%%%%%%%%%%%%%


%% Input Text Files %%%%%%%%%%%%%%%%%%%%%%%%%%%%%%%%%%%%%%%%%%%%%%%%%%%
%\usepackage[dvipsnames]{xcolor}
\usepackage{fancyvrb}

% redefine \VerbatimInput
\RecustomVerbatimCommand{\VerbatimInput}{VerbatimInput}%
	{fontsize=\footnotesize,
	%
	frame=lines,  % top and bottom rule only
	framesep=2em, % separation between frame and text
	rulecolor=\color{gray},
	%
	%label=\fbox{\color{black}data.txt},
	labelposition=topline,
	%
	commandchars=\|\(\), % escape character and argument delimiters for commands within the verbatim
	commentchar=*        % comment character
}
%%%%%%%%%%%%%%%%%%%%%%%%%%%%%%%%%%%%%%%%%%%%%%%%%%%%%%%%%%%%%%%%%%%%%%%


%% Bibliography %%%%%%%%%%%%%%%%%%%%%%%%%%%%%%%%%%%%%%%%%%%%%%%%%%%%%%%
\usepackage[numbers,square]{natbib}
\bibliographystyle{alphadin}
%\bibliographystyle{alpha}
%\bibliographystyle{siam}
%\bibliographystyle{apalike}
%\bibliographystyle{acm}    % [1], APP, N, alfabetico
%\bibliographystyle{phjcp}  % [1], N. APP, aparicion
%%%%%%%%%%%%%%%%%%%%%%%%%%%%%%%%%%%%%%%%%%%%%%%%%%%%%%%%%%%%%%%%%%%%%%%


%% TODOs %%%%%%%%%%%%%%%%%%%%%%%%%%%%%%%%%%%%%%%%%%%%%%%%%%%%%%%%%%%%%%
%\usepackage{pdfcomment}
\usepackage[%
	%disable,
	english, %LANGUAGE
	%german, %LANGUAGE
	textsize=tiny,
	colorinlistoftodos
]{todonotes}
\newcommand{\todoref}[1]{\todo[color=blue!40]{Missing Reference #1}}
\newcommand{\todoedit}[1]{\todo[color=green!40]{#1}}
%%%%%%%%%%%%%%%%%%%%%%%%%%%%%%%%%%%%%%%%%%%%%%%%%%%%%%%%%%%%%%%%%%%%%%%


%% HyperRef %%%%%%%%%%%%%%%%%%%%%%%%%%%%%%%%%%%%%%%%%%%%%%%%%%%%%%%%%%%
% http://www.tug.org/applications/hyperref/manual.html
\usepackage[
	pdftitle={\thesistitle},	% Sets the document information Title field
	pdfauthor={\autor},			% Sets the document information Author field
	pdfsubject={\thesistype},	% Sets the document information Subject field
	draft=false,				% Overwrites documentclass draft option 
	backref=section,			% Adds backlink text to the end of each item in the bibliography, as a list of section numbers
	pdfpagelabels,
	pdfpagelayout=TwoColumnRight,
	pdfdisplaydoctitle=true,	% Display document title instead of file name in title bar
	hypertexnames=false,		% To properly display the bookmarks
	%linktocpage 				% Social bookmarking numbers instead of text in the table of contents
	bookmarks,
	bookmarksnumbered=true,
	bookmarksopen=true,
	%bookmarksopenlevel=1,
	hyperfootnotes=false,
	hyperfigures=true,
	%unicode=true,
	colorlinks=true,
	%% These color definitions should be used for printing (all black)
	%linkcolor=black,		% Simple internal links
	%anchorcolor=black,		% Anchor text
	%citecolor=black,		% References to the bibliography entries in text
	%filecolor=black,		% Shortcuts, open local files
	%menucolor=black,		% Acrobat menu items
	%urlcolor=black,		% Color of external URLs linked text
	%% These color definitions can be used for distribution in PDF or for colored printing
	linkcolor=darkblue,		% Simple internal links
	citecolor=darkblue,		% References to the bibliography entries in text
	menucolor=darkblue,		% Acrobat menu items
	urlcolor=cyan,			% Color of external URLs linked text
]{hyperref}

% Link URL, long URLs wrap etc.
\usepackage{url}
%%%%%%%%%%%%%%%%%%%%%%%%%%%%%%%%%%%%%%%%%%%%%%%%%%%%%%%%%%%%%%%%%%%%%%%


%% Paragraph Settings %%%%%%%%%%%%%%%%%%%%%%%%%%%%%%%%%%%%%%%%%%%%%%%%%
%\setlength{\parskip}{3pt}
%\setlength{\parindent}{0pt}
%%%%%%%%%%%%%%%%%%%%%%%%%%%%%%%%%%%%%%%%%%%%%%%%%%%%%%%%%%%%%%%%%%%%%%%


%% Abbreviations %%%%%%%%%%%%%%%%%%%%%%%%%%%%%%%%%%%%%%%%%%%%%%%%%%%%%%
\usepackage[intoc]{nomencl}
\let\abbrev\nomenclature
%\renewcommand{\nomname}{Abkürzungsverzeichnis} %LANGUAGE
\renewcommand{\nomname}{Abbreviations} %LANGUAGE
\setlength{\nomlabelwidth}{.25\hsize}
\renewcommand{\nomlabel}[1]{#1 \dotfill} % dots between abbreviations and explainations
\setlength{\nomitemsep}{-\parsep} % Decrease line spacing
%%%%%%%%%%%%%%%%%%%%%%%%%%%%%%%%%%%%%%%%%%%%%%%%%%%%%%%%%%%%%%%%%%%%%%%


%%%%%%%%%%%%%%%%%%%%%%%%%%%%%%%%%%%%%%%%%%%%%%%%%%%%%%%%%%%%%%%%%%%%%%%
\usepackage{siunitx}
\sisetup{
	%locale=UK, %LANGUAGE
	%locale=US, %LANGUAGE
	locale=DE,  %LANGUAGE
	load-configurations=binary,
	load-configurations=abbreviations,
	per-mode=symbol,
}
\DeclareSIUnit\bit{Bit}
\DeclareSIUnit\byte{Byte}
% Usage:
%	\SI[options]{value}[pre-unit]{unit}
%	\si[options]{unit}
%	\num[options]{number}
%	\ang[options]{angle}
%%%%%%%%%%%%%%%%%%%%%%%%%%%%%%%%%%%%%%%%%%%%%%%%%%%%%%%%%%%%%%%%%%%%%%%


%% more featurerich and modern tables %%%%%%%%%%%%%%%%%%%%%%%%%%%%%%%%%
\usepackage{booktabs}
\setlength{\belowbottomsep}{\belowrulesep} % replaces 0pt-distance between bottomrule and caption
%\renewcommand*{\arraystretch}{1.2} more space between rows
% Usage:
%   \toprule, \midrule, \bottomrule
%   \cmidrule, \addlinespace
%%%%%%%%%%%%%%%%%%%%%%%%%%%%%%%%%%%%%%%%%%%%%%%%%%%%%%%%%%%%%%%%%%%%%%%


%% Miscellaneous %%%%%%%%%%%%%%%%%%%%%%%%%%%%%%%%%%%%%%%%%%%%%%%%%%%%%%
\usepackage{ifdraft}					% run depending on draft or final in documentclass % \ifdraft{draft case}{final case}
\usepackage{amsmath,amssymb,amstext} 	% For mathematical symbols
\usepackage{upgreek}					% For non-italic Greek letters, eg. \ Uppi
%\usepackage{array}
\usepackage{xspace}
\usepackage{xcolor}
\usepackage{multirow}
\usepackage{multicol}
\usepackage{scrhack}					% To prevent a listing Warning
\usepackage{listings}					% Program Code
%\usepackage{chngcntr}					% Continuous renumber the footnotes

\usepackage{caption}
	\captionsetup{justification=raggedright,format=hang,margin=30pt,font=small,labelfont=bf,labelsep=endash}

\usepackage{paralist}
\usepackage[yyyymmdd,hhmmss]{datetime}
%%%%%%%%%%%%%%%%%%%%%%%%%%%%%%%%%%%%%%%%%%%%%%%%%%%%%%%%%%%%%%%%%%%%%%%


%%%%%%%%%%%%%%%%%%%%%%%%%%%%%%%%%%%%%%%%%%%%%%%%%%%%%%%%%%%%%%%%%%%%%%%
%%%%%%%%%%%%%%%%%%%%%%%%%%%%%%%%%%%%%%%%%%%%%%%%%%%%%%%%%%%%%%%%%%%%%%%
\makenomenclature


%% Miscellaneous %%%%%%%%%%%%%%%%%%%%%%%%%%%%%%%%%%%%%%%%%%%%%%%%%%%%%%
\frenchspacing % generated a little more space behind a point

% Avoid widows and orphans (https://en.wikipedia.org/wiki/Widows_and_orphans http://de.wikipedia.org/wiki/Hurenkind)
\clubpenalty = 10000
\widowpenalty = 10000
\displaywidowpenalty = 10000

% Renumber footnotes consecutively
%\counterwithout{footnote}{chapter}

% Colors for Listings
\definecolor{darkblue}{rgb}{0,0,.5}
\definecolor{lightyellow}{rgb}{1,1,0.9}
\definecolor{lightgrey}{rgb}{0.95,0.95,0.95}
\definecolor{colKeys}{rgb}{0,0,1}
\definecolor{colIdentifier}{rgb}{0,0,0}
%\definecolor{colComments}{rgb}{1,0,0}		% red
%\definecolor{colComments}{rgb}{0.1,0.5,0}	% green
\definecolor{colComments}{rgb}{0.5,0.5,0.5}	% gray
\definecolor{colString}{rgb}{0,0.5,0}

% Formatting Source Edition
\renewcommand{\lstlistlistingname}{List of Source Code} %LANGUAGE
\renewcommand{\lstlistingname}{Source Code}
\lstset{
	float=tbph,								% makes sense on individual displayed listings only and lets them float
	language=C++,		  					% choose the language of the code
	basicstyle=\footnotesize,				% the size of the fonts that are used for the code
	numbers=left,							% where to put the line-numbers
	numberstyle=\tiny,						% the size of the fonts that are used for the line-numbers
	%numberstyle=\footnotesize,				% the size of the fonts that are used for the line-numbers
	stepnumber=1,							% the step between two line-numbers. If it's 1 each line will be numbered
	numbersep=7pt,							% how far the line-numbers are from the code
	backgroundcolor=\color{lightyellow},	% choose the background color. You must add \usepackage{color}
	showspaces=false,						% show spaces adding particular underscores
	showstringspaces=false,					% underline spaces within strings
	showtabs=false,							% show tabs within strings adding particular underscores
	frame=single,							% adds a frame around the code
	%frame={trLb},							% adds a frame around the code
	tabsize=4,								% sets default tabsize to 2 spaces
	captionpos=b,							% sets the caption-position (b or t)
	breaklines=true,						% sets automatic line breaking
	breakatwhitespace=false,				% sets if automatic breaks should only happen at whitespace
	breakindent=5pt,						% is the indention of the second, third,... line of broken lines
	%escapeinside={\%*}{*)},				% if you want to add a comment within your code
	identifierstyle=\color{colIdentifier},
	keywordstyle=\color{colKeys},
	stringstyle=\color{colString},
	commentstyle=\color{colComments},
	columns=fixed,
	%columns=flexible,
	%emph={bool,int,unsigned,char,true,false,void}, emphstyle=\color{blue},
	%emph={[2]\#include,\#define,\#ifdef,\#endif}, emphstyle={[2]\color{darkblue}},
}

% recalculation of the type area (to avoid overfull boxes)
%\KOMAoptions{DIV=last}

%%%%%%%%%%%%%%%%%%%%%%%%%%%%%%%%%%%%%%%%%%%%%%%%%%%%%%%%%%%%%%%%%%%%%%%

%% user defined commands and helpers %%%%%%%%%%%%%%%%%%%%%%%%%%%%%%%%%%

% abbreviations with correct spacing %LANGUAGE
%\newcommand{\ua}{\mbox{u.\,a.\ }}
\newcommand{\zB}{\mbox{z.\,B.\ }}
%\newcommand{\dahe}{\mbox{d.\,h.\ }}
\newcommand{\iic}{\mbox{I\texttwosuperior C}} % I²C

%%%%%%%%%%%%%%%%%%%%%%%%%%%%%%%%%%%%%%%%%%%%%%%%%%%%%%%%%%%%%%%%%%%%%%%
%% SSE Thesis Document Template (Thomas Dietrich) %%%%%%%%%%%%%%%%%%%%%
%%%%%%%%%%%%%%%%%%%%%%%%%%%%%%%%%%%%%%%%%%%%%%%%%%%%%%%%%%%%%%%%%%%%%%%
